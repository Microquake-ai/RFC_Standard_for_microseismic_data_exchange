\batchmode
\documentclass[12pt]{article}
\RequirePackage{ifthen}




\usepackage[letterpaper, margin=1in]{geometry}
\usepackage{enumitem}
\usepackage{titlesec}
\usepackage{xcolor}
\usepackage{graphicx}
\usepackage{siunitx}
\usepackage{upgreek}
\usepackage{natbib}
\usepackage{xspace}
\usepackage{longtable}
\usepackage{multirow}
\usepackage{mdframed}
\usepackage{minted}
\usepackage[acronym]{glossaries}
\usepackage[  colorlinks=true,           linkcolor=blue,            citecolor=green,           filecolor=magenta,         urlcolor=cyan            ]{hyperref}


\hypersetup{
    citecolor=blue
}


\usepackage{parskip} 

\setlength{\parindent}{0pt}  % Removes first-line indent from paragraphs

%
\providecommand{\museismic}{$\upmu$seismic\xspace}%
\providecommand{\muquake}{$\upmu$Quake\xspace} 


\DeclareSIUnit{\foot}{ft}


\makeglossaries


\newacronym{das}{DAS}{distributed acoustic sensing}
\newacronym{ai}{AI}{artificial intelligence}
\newacronym{fba}{FBA}{force balance accelerometer}
\newacronym{hdf5}{HDF5}{hierarchical data format version 5}
\newacronym{asdf}{ASDF}{adaptable seismic data format}
\newacronym{seed}{SEED}{standard for the exchange of earthquake data}
\newacronym{utc}{UTC}{coordinated universal time}
\newacronym{nrl}{NRL}{nominal response library}
\newacronym{smti}{SMTI}{seismic moment tensor inversion}
\newacronym{ppv}{PPV}{peak particle velocity}
\newacronym{ppa}{PPA}{peak particle acceleration}
\newacronym{ned}{NED}{north east down}
\newacronym{enu}{ENU}{east north up}








\makeatletter

\makeatletter
\count@=\the\catcode`\_ \catcode`\_=8 
\newenvironment{tex2html_wrap}{}{}%
\catcode`\<=12\catcode`\_=\count@
\newcommand{\providedcommand}[1]{\expandafter\providecommand\csname #1\endcsname}%
\newcommand{\renewedcommand}[1]{\expandafter\providecommand\csname #1\endcsname{}%
  \expandafter\renewcommand\csname #1\endcsname}%
\newcommand{\newedenvironment}[1]{\newenvironment{#1}{}{}\renewenvironment{#1}}%
\let\newedcommand\renewedcommand
\let\renewedenvironment\newedenvironment
\makeatother
\let\mathon=$
\let\mathoff=$
\ifx\AtBeginDocument\undefined \newcommand{\AtBeginDocument}[1]{}\fi
\newbox\sizebox
\setlength{\hoffset}{0pt}\setlength{\voffset}{0pt}
\addtolength{\textheight}{\footskip}\setlength{\footskip}{0pt}
\addtolength{\textheight}{\topmargin}\setlength{\topmargin}{0pt}
\addtolength{\textheight}{\headheight}\setlength{\headheight}{0pt}
\addtolength{\textheight}{\headsep}\setlength{\headsep}{0pt}
\newwrite\lthtmlwrite
\makeatletter
\let\realnormalsize=\normalsize
\global\topskip=2sp
\def\preveqno{}\let\real@float=\@float \let\realend@float=\end@float
\def\@float{\let\@savefreelist\@freelist\real@float}
\def\liih@math{\ifmmode$\else\bad@math\fi}
\def\end@float{\realend@float\global\let\@freelist\@savefreelist}
\let\real@dbflt=\@dbflt \let\end@dblfloat=\end@float
\let\@largefloatcheck=\relax
\let\if@boxedmulticols=\iftrue
\def\@dbflt{\let\@savefreelist\@freelist\real@dbflt}
\def\adjustnormalsize{\def\normalsize{\mathsurround=0pt \realnormalsize
 \parindent=0pt\abovedisplayskip=0pt\belowdisplayskip=0pt}%
 \def\phantompar{\csname par\endcsname}\normalsize}%
\def\lthtmltypeout#1{{\let\protect\string \immediate\write\lthtmlwrite{#1}}}%
\usepackage[tightpage,active]{preview}
\PreviewBorder=1bp
\newbox\lthtmlPageBox
\newdimen\lthtmlCropMarkHeight
\newdimen\lthtmlCropMarkDepth
\long\def\lthtmlTightVBoxA#1{\def\lthtmllabel{#1}
    \setbox\lthtmlPageBox\vbox\bgroup\catcode`\_=8 }%
\long\def\lthtmlTightVBoxZ{\egroup
    \lthtmlCropMarkHeight=\ht\lthtmlPageBox \advance \lthtmlCropMarkHeight 6pt
    \lthtmlCropMarkDepth=\dp\lthtmlPageBox
    \lthtmltypeout{^^J:\lthtmllabel:lthtmlCropMarkHeight:=\the\lthtmlCropMarkHeight}%
    \lthtmltypeout{^^J:\lthtmllabel:lthtmlCropMarkDepth:=\the\lthtmlCropMarkDepth:1ex:=\the \dimexpr 1ex}%
    \begin{preview}\copy\lthtmlPageBox\end{preview}}%
\long\def\lthtmlTightFBoxA#1{\def\lthtmllabel{#1}%
    \adjustnormalsize\setbox\lthtmlPageBox=\vbox\bgroup\hbox\bgroup %
    \let\ifinner=\iffalse \let\)\liih@math %
    \bgroup\catcode`\_=8 }%
\long\def\lthtmlTightFBoxZ{\egroup\egroup
    \@next\next\@currlist{}{\def\next{\voidb@x}}%
    \expandafter\box\next\egroup %
    \lthtmlCropMarkHeight=\ht\lthtmlPageBox \advance \lthtmlCropMarkHeight 6pt
    \lthtmlCropMarkDepth=\dp\lthtmlPageBox
    \lthtmltypeout{^^J:\lthtmllabel:lthtmlCropMarkHeight:=\the\lthtmlCropMarkHeight}%
    \lthtmltypeout{^^J:\lthtmllabel:lthtmlCropMarkDepth:=\the\lthtmlCropMarkDepth:1ex:=\the \dimexpr 1ex}%
    \begin{preview}\copy\lthtmlPageBox\end{preview}}%
    \long\def\lthtmlinlinemathA#1#2\lthtmlindisplaymathZ{\lthtmlTightVBoxA{#1}{\hbox\bgroup#2\egroup}\lthtmlTightVBoxZ}
    \def\lthtmlinlineA#1#2\lthtmlinlineZ{\lthtmlTightVBoxA{#1}{\hbox\bgroup#2\egroup}\lthtmlTightVBoxZ}
    \long\def\lthtmldisplayA#1#2\lthtmldisplayZ{\lthtmlTightVBoxA{#1}{#2}\lthtmlTightVBoxZ}
    \long\def\lthtmldisplayB#1#2\lthtmldisplayZ{\\edef\preveqno{(\theequation)}%
        \lthtmlTightVBoxA{#1}{\let\@eqnnum\relax#2}\lthtmlTightVBoxZ}
    \long\def\lthtmlfigureA#1{\let\@savefreelist\@freelist
        \lthtmlTightFBoxA{#1}}
    \long\def\lthtmlfigureZ{
        \lthtmlTightFBoxZ\global\let\@freelist\@savefreelist}
    \long\def\lthtmlpictureA#1{\let\@savefreelist\@freelist
        \lthtmlTightVBoxA{#1}}
    \long\def\lthtmlpictureZ{
        \lthtmlTightVBoxZ\global\let\@freelist\@savefreelist}
\def\lthtmlcheckvsize{\ifdim\ht\sizebox<\vsize 
  \ifdim\wd\sizebox<\hsize\expandafter\hfill\fi \expandafter\vfill
  \else\expandafter\vss\fi}%
\providecommand{\selectlanguage}[1]{}%
\makeatletter \tracingstats = 1 


\begin{document}
\pagestyle{empty}\thispagestyle{empty}\lthtmltypeout{}%
\lthtmltypeout{latex2htmlLength hsize=\the\hsize}\lthtmltypeout{}%
\lthtmltypeout{latex2htmlLength vsize=\the\vsize}\lthtmltypeout{}%
\lthtmltypeout{latex2htmlLength hoffset=\the\hoffset}\lthtmltypeout{}%
\lthtmltypeout{latex2htmlLength voffset=\the\voffset}\lthtmltypeout{}%
\lthtmltypeout{latex2htmlLength topmargin=\the\topmargin}\lthtmltypeout{}%
\lthtmltypeout{latex2htmlLength topskip=\the\topskip}\lthtmltypeout{}%
\lthtmltypeout{latex2htmlLength headheight=\the\headheight}\lthtmltypeout{}%
\lthtmltypeout{latex2htmlLength headsep=\the\headsep}\lthtmltypeout{}%
\lthtmltypeout{latex2htmlLength parskip=\the\parskip}\lthtmltypeout{}%
\lthtmltypeout{latex2htmlLength oddsidemargin=\the\oddsidemargin}\lthtmltypeout{}%
\makeatletter
\if@twoside\lthtmltypeout{latex2htmlLength evensidemargin=\the\evensidemargin}%
\else\lthtmltypeout{latex2htmlLength evensidemargin=\the\oddsidemargin}\fi%
\lthtmltypeout{}%
\makeatother
\setcounter{page}{1}
\onecolumn

% !!! IMAGES START HERE !!!

{\newpage\clearpage
\lthtmlinlinemathA{tex2html_wrap_inline360}%
$\upmu$%
\lthtmlindisplaymathZ
\lthtmlcheckvsize\clearpage}

\stepcounter{section}
\stepcounter{subsection}
\stepcounter{subsection}
\stepcounter{subsection}
\stepcounter{subsection}
\stepcounter{subsubsection}
\stepcounter{section}
\stepcounter{section}
\stepcounter{subsection}
\stepcounter{section}
\stepcounter{subsection}
\stepcounter{section}
\stepcounter{subsection}
\stepcounter{subsection}
{\newpage\clearpage
\lthtmlinlinemathA{tex2html_wrap_inline342}%
\( M_0 \)%
\lthtmlindisplaymathZ
\lthtmlcheckvsize\clearpage}

{\newpage\clearpage
\lthtmlinlinemathA{tex2html_wrap_inline343}%
\( f_c \)%
\lthtmlindisplaymathZ
\lthtmlcheckvsize\clearpage}

\stepcounter{subsection}
\stepcounter{subsection}
\stepcounter{section}
\stepcounter{subsection}
{\newpage\clearpage
\lthtmlfigureA{mdframed181}%
\begin{mdframed}
% latex2html id marker 181
[backgroundcolor=gray!20]
\textbf{Note on Coordinate Systems:} Utilizing a Cartesian coordinate system (x, y, z) introduces inherent ambiguities. Notably, the orientation of the x and y axes could align either with the North and East directions or be reversed. Additionally, the direction of the z-axis could be either upwards or downwards. To mitigate these ambiguities, we recommend imposing constraints on the alignment of the x, y, and z axes to ensure the coordinate system is right-handed and aligned with geographical axes. Specifically, we advocate the use of \gls{enu} and \gls{ned} coordinate systems.
\par
$\upmu$Quake\xspace version 2.0 introduces the \texttt{Coordinates} Class and implements the management of the Cartesian coordinate system in the relevant Obspy objects, allowing the integration of the information in the QuakeML and StationXML format (see Appendix~\ref{app:coordinate_system_handling}). The handling is done by converting the \texttt{Coordinate} object to JSON and writing the JSON string as an object extra parameters. The extra parameters are then persisted to the QuakeML or StationXML by creating a new namespace inside the QuakeML or StationXML file. 
\par
\begin{verbatim}

<?xml version='1.0' encoding='utf-8'?>
    <q:quakeml xmlns:mq="https://microquake.ai/xml/event/1" 
    xmlns="http://quakeml.org/xmlns/bed/1.2" 
    xmlns:q="http://quakeml.org/xmlns/quakeml/1.2">\end{verbatim}

\par
\end{mdframed}%
\lthtmlfigureZ
\lthtmlcheckvsize\clearpage}

\stepcounter{subsection}
\stepcounter{subsubsection}
{\newpage\clearpage
\lthtmlinlinemathA{tex2html_wrap_inline344}%
\(f_0\)%
\lthtmlindisplaymathZ
\lthtmlcheckvsize\clearpage}

{\newpage\clearpage
\lthtmlinlinemathA{tex2html_wrap_inline345}%
\(E_p\)%
\lthtmlindisplaymathZ
\lthtmlcheckvsize\clearpage}

{\newpage\clearpage
\lthtmlinlinemathA{tex2html_wrap_inline346}%
\(P\)%
\lthtmlindisplaymathZ
\lthtmlcheckvsize\clearpage}

{\newpage\clearpage
\lthtmlinlinemathA{tex2html_wrap_inline347}%
\(E_s\)%
\lthtmlindisplaymathZ
\lthtmlcheckvsize\clearpage}

{\newpage\clearpage
\lthtmlinlinemathA{tex2html_wrap_inline348}%
\(S\)%
\lthtmlindisplaymathZ
\lthtmlcheckvsize\clearpage}

{\newpage\clearpage
\lthtmlinlinemathA{tex2html_wrap_inline398}%
$f_0$%
\lthtmlindisplaymathZ
\lthtmlcheckvsize\clearpage}

{\newpage\clearpage
\lthtmlinlinemathA{tex2html_wrap_inline400}%
$E_p$%
\lthtmlindisplaymathZ
\lthtmlcheckvsize\clearpage}

{\newpage\clearpage
\lthtmlinlinemathA{tex2html_wrap_inline402}%
$E_s$%
\lthtmlindisplaymathZ
\lthtmlcheckvsize\clearpage}

{\newpage\clearpage
\lthtmlinlinemathA{tex2html_wrap_inline410}%
$M_0$%
\lthtmlindisplaymathZ
\lthtmlcheckvsize\clearpage}

\stepcounter{paragraph}
\stepcounter{subsection}
\stepcounter{subsubsection}
\stepcounter{paragraph}
\stepcounter{subsubsection}
\stepcounter{subsubsection}
\stepcounter{subsection}
\stepcounter{subsubsection}
\stepcounter{subsubsection}
\stepcounter{section}
\stepcounter{subsection}
\stepcounter{subsubsection}
\stepcounter{subsection}
\stepcounter{section}
\appendix
\stepcounter{section}

\renewcommand{\arraystretch}{2}
{\newpage\clearpage
\lthtmlinlinemathA{tex2html_wrap_inline793}%
$M_w = \frac{2}{3} \left( \log_{10} M_0 - 9.1 \right)$%
\lthtmlindisplaymathZ
\lthtmlcheckvsize\clearpage}

{\newpage\clearpage
\lthtmlinlinemathA{tex2html_wrap_inline795}%
$E = E_p + E_s$%
\lthtmlindisplaymathZ
\lthtmlcheckvsize\clearpage}

{\newpage\clearpage
\lthtmlinlinemathA{tex2html_wrap_inline797}%
$R = \left(\frac{3M_0}{4\pi\mu}\right)^{1/3}$%
\lthtmlindisplaymathZ
\lthtmlcheckvsize\clearpage}

{\newpage\clearpage
\lthtmlinlinemathA{tex2html_wrap_inline799}%
$P = \frac{M_0}{\mu}$%
\lthtmlindisplaymathZ
\lthtmlcheckvsize\clearpage}

{\newpage\clearpage
\lthtmlinlinemathA{tex2html_wrap_inline801}%
$\sigma_a = \frac{E}{M_0}$%
\lthtmlindisplaymathZ
\lthtmlcheckvsize\clearpage}

{\newpage\clearpage
\lthtmlinlinemathA{tex2html_wrap_inline803}%
$V_a = \frac{E}{\rho V_p^2}$%
\lthtmlindisplaymathZ
\lthtmlcheckvsize\clearpage}

{\newpage\clearpage
\lthtmlinlinemathA{tex2html_wrap_inline805}%
$\Delta\sigma_{\text{stat}} = \frac{7}{16} \left(\frac{M_0}{\pi R^3}\right)^{1/3}$%
\lthtmlindisplaymathZ
\lthtmlcheckvsize\clearpage}

{\newpage\clearpage
\lthtmlinlinemathA{tex2html_wrap_inline807}%
$\Delta\sigma_{\text{dyn}} = \frac{7}{16} \rho V_s^2 \left(\frac{M_0}{\pi R^3}\right)^{1/3}$%
\lthtmlindisplaymathZ
\lthtmlcheckvsize\clearpage}

{\newpage\clearpage
\lthtmlinlinemathA{tex2html_wrap_inline809}%
$\rho$%
\lthtmlindisplaymathZ
\lthtmlcheckvsize\clearpage}

{\newpage\clearpage
\lthtmlinlinemathA{tex2html_wrap_inline811}%
$V_p$%
\lthtmlindisplaymathZ
\lthtmlcheckvsize\clearpage}

{\newpage\clearpage
\lthtmlinlinemathA{tex2html_wrap_inline813}%
$V_s$%
\lthtmlindisplaymathZ
\lthtmlcheckvsize\clearpage}

{\newpage\clearpage
\lthtmlinlinemathA{tex2html_wrap_inline815}%
$R$%
\lthtmlindisplaymathZ
\lthtmlcheckvsize\clearpage}

{\newpage\clearpage
\lthtmlinlinemathA{tex2html_wrap_inline817}%
$\mu$%
\lthtmlindisplaymathZ
\lthtmlcheckvsize\clearpage}

\stepcounter{section}
{\newpage\clearpage
\lthtmlinlinemathA{tex2html_wrap_inline849}%
\( \pm 2.5 \)%
\lthtmlindisplaymathZ
\lthtmlcheckvsize\clearpage}

{\newpage\clearpage
\lthtmlfigureA{minted834}%
\begin{minted}[frame=lines, framesep=2mm]{python} label=lst:uquake_example]
import numpy as np
from uquake.core import Stream, Trace, UTCDateTime
from uquake.core.inventory import Inventory, Network, Station, Channel, Site
from uquake.core.inventory.response import Response, PolesZerosResponseStage, CoefficientsTypeResponseStage
\par
# Create Inventory, Network, and Station as before
inv = Inventory(networks=[], source="Example")
net = Network(code="XX", stations=[], description="Example Network")
sta = Station(code="STA1", x=0.0, y=0.0, z=0.0, site=Site(name="Example Site"))
\par
# Create the Channel and Response objects
cha = Channel(code="HHZ", location_code="", x=0.0, y=0.0, z=0.0, sample_rate=100.0)
resp = Response()
\par
# Poles and Zeros for 15 Hz geophone
pz_stage = PolesZerosResponseStage(
    stage_sequence_number=1,
    stage_gain=1.0,
    stage_gain_frequency=1.0,
    input_units="M/S",
    output_units="V",
    pz_transfer_function_type="LAPLACE (RADIANS/SECOND)",
    normalization_frequency=1.0,
    zeros=[0j],
    poles=[-94.44j, 94.44j],
    normalization_factor=1.0
)
\par
# Coefficients for 24-bit digitizer
coeff_stage = CoefficientsTypeResponseStage(
    stage_sequence_number=2,
    stage_gain=1 / (2.5 / (2**23)),
    stage_gain_frequency=1.0,
    input_units="V",
    output_units="COUNT",
    cf_transfer_function_type="DIGITAL",
    numerator=[1.0],
    denominator=[]
)
\par
# Add stages and complete the hierarchy
resp.response_stages.append(pz_stage)
resp.response_stages.append(coeff_stage)
cha.response = resp
sta.channels.append(cha)
net.stations.append(sta)
inv.networks.append(net)
\par
# Create a Stream with random ADC count values simulating the data
npts = 1000
starttime = UTCDateTime(0)
sampling_rate = 100.0
\par
# Create a single Trace object
trace = Trace(data=np.random.randint(-2**23, 2**23, npts))
trace.stats.starttime = starttime
trace.stats.sampling_rate = sampling_rate
trace.stats.network = "XX"
trace.stats.station = "STA1"
trace.stats.channel = "HHZ"
\par
# Create a Stream object and append the Trace
stream = Stream(traces=[trace])
\par
# Attach the response to the Stream
stream.attach_response(inv)
\par
# Convert waveform from ADC count to physical units
stream.remove_response(output="ACC")
stream.remove_response(output="VEL")
stream.remove_response(output="DISP")
\end{minted}%
\lthtmlfigureZ
\lthtmlcheckvsize\clearpage}

\stepcounter{subsection}
{\newpage\clearpage
\lthtmlinlinemathA{tex2html_wrap_inline859}%
$\Omega$%
\lthtmlindisplaymathZ
\lthtmlcheckvsize\clearpage}

{\newpage\clearpage
\lthtmlinlinemathA{tex2html_wrap_inline850}%
\(\tau = RC\)%
\lthtmlindisplaymathZ
\lthtmlcheckvsize\clearpage}

{\newpage\clearpage
\lthtmlinlinemathA{tex2html_wrap_indisplay1376}%
$\displaystyle \tau = 2500 \Omega \times 10 \times 10^{-9} F = 25 \times 10^{-6} s
$%
\lthtmlindisplaymathZ
\lthtmlcheckvsize\clearpage}

{\newpage\clearpage
\lthtmlfigureA{minted840}%
\begin{minted}[frame=lines]{python}
from uquake.core.inventory import (Response, Coefficients, Polynomial, 
                                   PolesZeros, Blockette)
\par
# Sensor response already created
response_sensor = ...
\par
# Create the cable stage
cable_tau = 25e-6  # in seconds
cable_pole_zero = PolesZeros(
    pz_transfer_function_type="LAPLACE (RADIANS/SECOND)",
    normalization_frequency=0.0,
    zeros=[0.0],
    poles=[-1 / cable_tau],
    normalization_factor=1.0,
)
\par
# Create the cable response stage
cable_stage = Blockette(
    stage_sequence_number=2,
    response=cable_pole_zero
)
\par
# Add the cable stage to the existing response
response_sensor.response_stages.append(cable_stage)
\par
# Add this to your channel
channel.response = response_sensor
\end{minted}%
\lthtmlfigureZ
\lthtmlcheckvsize\clearpage}

\stepcounter{subsection}
{\newpage\clearpage
\lthtmlfigureA{minted845}%
\begin{minted}[frame=lines]{python}
# Using uquake
from uquake.core.inventory import Inventory, Network, Station, Channel, Site
from uquake.clients.nrl import NRL
\par
# Initialize NRL client
nrl = NRL()
\par
# Construct response from NRL
response_uquake = nrl.get_response(
    sensor_keys=['Sensor Manufacturer', 'Sensor Model'],
    datalogger_keys=['Datalogger Manufacturer', 'Datalogger Model']
)
\par
# Define channel with x, y, z coordinates
channel = Channel(
    code="EHZ",
    location_code="",
    x=0.0,
    y=0.0,
    z=0.0,
    response=response_uquake
)
\end{minted}%
\lthtmlfigureZ
\lthtmlcheckvsize\clearpage}

\stepcounter{section}
\stepcounter{subsection}
\stepcounter{subsection}
{\newpage\clearpage
\lthtmlinlinemathA{tex2html_wrap_inline961}%
\((dx, dy, dz)\)%
\lthtmlindisplaymathZ
\lthtmlcheckvsize\clearpage}

\stepcounter{subsection}

\end{document}
