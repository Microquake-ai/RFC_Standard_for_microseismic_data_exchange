\subsection{Stream Naming}

We suggest adopting a relaxed version of the \gls{seed} Identifier Convention compatible with the QuakeML and StationXML standards. The StationXML does not restrict the string length; QuakeML does. We therefore suggest adopting the convention presented in section 3.3.5 of the QuakeML Version 1.2 (revision 20130214b) (see \href{https://quake.ethz.ch/quakeml/docs/latest?action=AttachFile&do=get&target=QuakeML-BED.pdf}{QuakeML Reference Manual}). This convention provides a standardized approach to uniquely identify seismic data streams by anchoring them to their source network, station, location, and channel, ensuring clarity and consistency in data identification and handling across various seismic data systems.

The QuakeML naming convention comprises four distinct parts that are:

\begin{description}
  \item[Network] Represents a collection of stations grouped by a specific monitoring objective or designed to target a distinct area within a mine. This grouping may not necessarily reflect physical proximity but is often organized based on monitoring needs or operational considerations.
  \item[Station] Within a given network, the term station denotes a coordinated set of instruments, often clustered together based on their monitoring function or spatial considerations. For example, a station can comprise a set of instruments installed in a single borehole or deployed from the same location in boreholes of different orientations.
  \item[Location] Specifies an individual instrument within a station, allowing differentiation when multiple instruments exist at the same station. For example, in a borehole where multiple instruments are installed, each instrument would refer to a location.
  \item[Channel] Details the specific recording component or the type of measurement undertaken by the instrument at the given location.
\end{description}

A unique stream name is obtained by concatenating those parts together as follows:
\begin{center}
\texttt{NETWORK.STATION.LOCATION.CHANNEL}
\end{center}





% \begin{description}
%     \item[Network] -- a network represents an organized collection of seismic sensors or stations, deployed to monitor microseismic activity within a specific region or setting. In the context of underground mining, a single mine can host multiple networks, each designed to address distinct monitoring objectives or cover specific sections of the mine. These networks collectively provide comprehensive spatial coverage and cater to various monitoring resolutions and sensitivities, ensuring a robust understanding of the mine's seismicity.
%     \item[Station] -- a station refers to a specific site or physical location where one or more seismic sensors or instruments are deployed. A station aggregates the data from its sensors and serves as a key reference point for spatially locating seismic events within the monitoring network.
%     \item[Location] -- a finer granularity within a station, a location delineates a specific subset of instruments or sensors within that station, potentially positioned at different depths, azimuths, or other spatial orientations. Locations allow for a detailed spatial representation of sensors, especially critical in complex settings like mines where multiple sensors might be co-located within a single station but at different orientations or depths.
%     \item[Channel] -- a channel describes a singular data stream produced by a specific sensor or instrument. Channels capture distinct measurements like vertical motion, horizontal motion in particular directions, or other seismic attributes. Each channel is associated with unique calibration and response characteristics, ensuring the accurate representation of the seismic waves as detected by the sensor.
% \end{description}



