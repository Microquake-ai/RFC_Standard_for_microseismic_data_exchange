\subsection{General Format Considerations}

This section delves into the overarching design principles and foundational elements underpinning the proposed format, ensuring adherence to the objectives of seamless, lossless, and convenient data interchange across platforms.

\begin{enumerate}
    \item \textbf{Data Integrity and Consistency}: Details on mechanisms, potentially including checksums, versioning, or other measures, to ensure data remains unaltered and consistent during transfers.
    
    \item \textbf{Platform Independence}: Insights into how the format maintains neutrality to specific proprietary platforms, endorsing cross-platform compatibility.
    
    \item \textbf{Scalability and Flexibility}: Addressing the format's ability to manage both small and expansive datasets, e.g., high-density \gls{das} data.
    
    \item \textbf{Usability}: Features enhancing the user-friendliness of the format, which may encompass aspects like readability, annotations, or metadata.
    
    \item \textbf{Efficiency}: Discussion around computational considerations, ensuring both data storage and exchange remain efficient.
    
    \item \textbf{Extensibility}: Considerations on the format's design, allowing it to evolve and accommodate future technological shifts or data needs.
    
    \item \textbf{Standardized References}: Incorporating standardized naming conventions, directory structures, and file naming conventions.
    
    \item \textbf{Interoperability with Existing Formats}: Examination of the format's relationship or interaction with prevailing data formats in the domain.
\end{enumerate}
